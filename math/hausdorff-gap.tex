---
title: Hausdorff Gap の証明
author: 石井 大海
tag: 数学,数理論理学,集合論,無限組合せ論
description: 集合論における Hausdorff Gap の証明.
date: 2013/09/01 23:38:34 JST
---

\documentclass[a4j,uplatex]{jsarticle}
\usepackage{amsmath,amssymb}
\usepackage{epstopdf}
\usepackage{graphicx}
\DeclareGraphicsRule{.tif}{png}{.png}{`convert #1 `dirname #1`/`basename #1 .tif`.png}
\bibliographystyle{jplain}
\usepackage{bm}
\usepackage{epstopdf}
\usepackage{cases}
\usepackage{enumerate}
\usepackage{mystyle}

\newcommand{\Fin}{\mathrm{Fin}}

\title{Hausdorff Gap の証明}
\author{石井大海}

\begin{document}
\maketitle

\begin{theorem}[Hausdorff]
 ブール代数 $\mathfrak{P}\omega / \Fin$ について,次を満たす $\{a_\alpha\}_{\alpha < \omega_1}, \{b_\alpha\}_{\alpha < \omega_1}$が存在する.
 \begin{enumerate}
  \item $a_\alpha < a_\beta < b_\beta < a_\alpha\; (\alpha < \beta < \omega_1)$
  \item $a_\alpha \leq b \leq b_\alpha \; (\alpha < \omega_1)$ を満たすような $b \in \mathfrak{P}\omega / \Fin$ は存在しない.
 \end{enumerate}
\end{theorem}

以後,$[\quad]: \mathfrak{P}\omega \to \mathfrak{P}\omega/\Fin$ を標準写像とする.この定理の証明の為に,幾つかの命題を証明しておく.

まず,次の事は簡単に確認出来る:

\begin{fact}
 \begin{enumerate}[(i)]
  \item $[A] = 1 \Leftrightarrow A \text{は補有限}$
  \item $[A] \leq [B] \Leftrightarrow A \setminus B \in \Fin$
  \item $[A] \neq [B] \Leftrightarrow A \vartriangle B \notin \Fin$
 \end{enumerate}
\end{fact}

\begin{lemma}\label{lem:01}
 $a_n \leq a_{n+1} < 1 \; (n < \omega)$ ならば,$a_n \leq b < 1 \; (n < \omega)$ となるような $b$ が存在する.
\end{lemma}
\begin{proof}
 $[A_n] = a_n$ となるような $A_n \subseteq \omega$ を取る.次のようにして,$j_n < j_{n+1}$ を $A_i \cap [j_n, \omega) \subseteq A_n \; (i < n)$ を満たすように再帰的に定める.

 まず,$j_0 = 0$ とする.そこで $j_n$ まで $A_i \cap [j_n, \omega) \subseteq A_n \; (i < n)$ を満たすように取れているとして,$j_{n+1}$ を作りたい.ここで,
 \[
  j_{n+1} = \min \Set{j_n < j < \omega | A_n \cap [j, \omega) \subseteq A_{n+1}}
 \]
 により $j_{n+1}$ を定めよう.もし右辺の集合が空集合であれば,どんな $j > j_n$ に対しても $A_n \cap [j, \omega) \subsetneq A_{n+1}$ となるので,$|A_n \setminus A_{n+1}| = \aleph_0$ となる.しかし,仮定より $[A_n] \leq [A_{n+1}]$ であったので,$A_n \setminus A_{n+1} \in \Fin$ でなくてはならず,矛盾.よって $n < \omega$ に対し常に題意を満たす $j_n$ が取れる.

 次に,$j_n \leq k_n, k_n < k_{n+1} \; (n < \omega)$ を満たすように $k_n \notin A_n$ を構成したい.今,仮定より $A_n$ は補有限ではないので,$\omega \setminus A_n$ は $\omega$ で非有界である.よって,このような $k_n$ は常に取れる.

 すると,$m > n$ なら $k_m \notin A_n$ が成立する.なぜなら,$m > n$ の時 $A_n \cap [j_m, \omega) \subseteq A_m$ であり,今 $j_m \leq k_m$ であったので $A_n \cap [k_m, \omega) \subseteq A_m$ である.ここで $k_m \in A_n$ とすると,$k_m \in A_n \cap [k_m, \omega) \subseteq A_m$ となるが,$k_m \in \omega \setminus A_m$ なので矛盾.

 そこで,$A = \omega \setminus \Set{k_n | n < \omega}$ とおけば,$b = [A]$ が求めるものである.まず,構成から $A$ は補有限でないので, $b = [A] < 1$ である.また,$A_n' = A_n \cap [k_n, \omega)$ とおけば,$A_n \setminus A_n' \subseteq [0, k_n) \in \Fin$ より $[A_n] \leq [A_n']$.また $A_n' \subset A_n$ より $[A_n] \leq [A_n']$.よって $[A_n] = [A_n'] = a_n$ である.$j \in A_n'$ とすると,$j > k_n$ かつ $j \neq k_m \; (m > n)$.よって,$A_n' \subseteq \omega \setminus \Set{ k_n | n < \omega }$となるので,$a_n = [A_n] \leq [A] = b$ である.
 \mbox{}
\end{proof}

\begin{lemma}\label{lem:02}
 $a_n \leq a_{n+1}, b_n \leq b_{n+1}, a_n \wedge b_n = 0 \; (n < \omega)$
 ならば,$a_n \leq c$ かつ $b_n \wedge c = 0 \; (n < \omega)$ となる $c$ が存在する.
\end{lemma}

\begin{proof}
 $a_n = [A_n], b_n = [B_n]$ とする.$a_n \wedge b_n = 0$ より $A_n \cap B_n \in \Fin \; (n < \omega)$ である.

 そこで,
 \begin{equation*}
  \begin{aligned}
   A_i \cap [j_n, \omega) \subseteq A_n\\
   A_n \cap B_i \subseteq [0, j_n)
  \end{aligned} \quad (i \leq n)  \tag{*}\label{def:jn01}
 \end{equation*}
 を満たすように $j_n < j_{n+1}$ を取りたい.まず,$A_n \cap B_n \in \Fin$ より,$A_0 \cap B_0 \subseteq [0, j_0)$ となるような最小の $j_0$ が取れる.この時,$A_0 \cap [j_0, \omega) \subseteq A_0$ は自明に成立しているので,$n = 0$ の時は OK.
 そこで,$(*)$ を満たす $j_n$ が取れているとき,$j_{n+1}$ を次のように定める:
 \[
  j_{n+1} = \min \Set{j_n < j < \omega | A_{n+1} \cap B_i \subseteq [0,j) \; (i < n + 1), A_n \cap [j, \omega) \subseteq A_{n+1}}
 \]
 ここで,$A_{n+1} \cap B_i \subseteq [0,j)$ となるような $j$ が取れることは補題 $1$ の証明で既に示した.また,$A_{n+1} \cap B_i \in \Fin\;(i \leq n+1)$ だから,各 $i$ に対し $\subseteq [0, j)$ となるような $j$ が取れる.全順序性より二条件を満たすものは明らかに存在するので,$j_{n+1}$ は well-defined である.以上から,$j_n < j_{n+1}$ が取れる.

 ここで $A_n' = A_n \cap [j, \omega)$ とおくと,有限の差しかないので $[A_n'] = [A_n] = a_n$ である.そこで,
 \[
  C = \bigcup \Set{ A_n' | n < \omega } = \bigcup \Set{A_n \cap [j_n, \omega) | n < \omega}
 \]
 として,$c = [C]$ とおけば,$a_n \leq c$ を満たす.また,
 \begin{align*}
  B_m \cap C &= \bigcup_{n < \omega} (A_n \cap B_m \cap [j_n, \omega))\\
  &= \bigcup_{n < m} \left(A_n \cap B_m \cap [j_n, \omega)\right) \cup \bigcup_{m \leq n < \omega} \left(A_n \cap B_m \cap [j_n, \omega)\right)\\
  &\subseteq \bigcup_{n < m} \left(A_n \cap B_m \cap [j_n, \omega)\right) \cup \bigcup_{m \leq n < \omega} \left([0, j_n) \cap [j_n, \omega)\right)\\
  &\subseteq \bigcup_{n < m} \left(A_n \cap B_m\right) \in \Fin
 \end{align*}
 よって,$b_m \wedge c = 0 \; (m < \omega)$ も成立.\mbox{}
\end{proof}

\begin{lemma}
 $\lambda$ を無限順序数とする.$X \subseteq \omega, X_\alpha \subseteq \omega \; (\alpha < \lambda), [X] \leq [Y]$ とする.このとき,もし任意の $k < \omega$ について $\Set{\alpha < \lambda | X_\alpha \cap X \subseteq k}$ が有限なら,$Y$ も同様の性質を満たす.
\end{lemma}
\begin{proof}
 対偶を示す.つまり,$[X] \leq [Y]$ として,ある $k < \omega$ に対し,$Y \cap X_{\alpha_j} \subseteq k \; (j < \omega)$ を満たすような $\alpha_j < \omega$ が取れたとする.今,$[X] \leq [Y]$ より $X \setminus Y \in \Fin$.そこで,$\ell = \sup^+(X \setminus Y) < \omega$ と置く.この時,
 \begin{align*}
  X \cap X_{\alpha_j} &= ((X \cap Y) \cup (X \setminus Y)) \cap X_{\alpha_j}\\
  &= (X \cap Y \cap X_{\alpha_j}) \cup (X \setminus Y) \cap X_{\alpha_j}\\
  &\subseteq k \cup \ell = \max(k, \ell) 
 \end{align*}
 よって $m = \max(k, \ell)$ とおけば $\Set{\alpha < \lambda | X \cap X_\alpha \subseteq m} \notin \Fin$ となる.よって示された.\mbox{}
\end{proof}

以上,三つの補題が,以下の証明において本質的な役割を果す.

\begin{proof}[定理の証明]
 以下を満たすように $A_\alpha, B_\alpha \; (\alpha < \omega_1)$ を帰納的に構成する:
 \begin{enumerate}[(a)]
  \item $[A_\alpha] \vee [B_\alpha] < 1$
	\label{cond:join-coinfinite}
  \item $[A_\alpha] \wedge [B_\alpha] = 0$
	\label{cond:almost-disj}
  \item $[A_\alpha] < [A_\beta], [B_\alpha] < [B_\beta]\;(\alpha < \beta < \omega_1)$
	\label{cond:increasing}
  \item 各 $k < \omega, \beta < \omega_1$ に対し,
	$\Set{\alpha < \beta | A_\beta \cap B_\alpha \subseteq k}$ は有限
	\label{cond:finiteness}
 \end{enumerate}

 $\alpha = 0$ の時は,$A_0 = B_0 = \emptyset$ とおけばよい.

 $\alpha$ が後続順序数の時.$A_{\alpha + 1}, B_{\alpha + 1}$ を作ることを考える.$\beta < \alpha$ とすると,帰納法の仮定より $A_\alpha \cup B_\beta$ は補有限ではない.そこで,$\omega \setminus (A_\alpha \cup B_\alpha) = \Set{ n_k | k < \omega } \; (\ell < k \Rightarrow n_\ell < n_k)$ として,
 \begin{gather*}
  P = \Set{n_k | k \equiv 0 \pmod{3}} \quad Q = \Set{n_k | k \equiv 1 \pmod{3}}\\
  A_{\alpha + 1} = A_\alpha \cup P \quad B_{\alpha + 1} = B_\alpha \cup Q
 \end{gather*}
  とおく.このとき,$\omega \setminus (A_{\alpha + 1} \cup B_{\alpha + 1}) = \Set{n_k | k \equiv 2 \pmod{3}}$ となるので,$[A_{\alpha + 1}] \vee [B_{\alpha+1}] < 1$ である.よって条件 $(a)$ は成立.また,条件$(b)$ についても,
 \begin{align*}
  A_{\alpha + 1} \cap B_{\alpha+1} &= (A_\alpha \cup P) \cap (B_\alpha \cup Q)\\
  &= (A_\alpha \cap B_\alpha) \cup \underbrace{ (A_\alpha \cap Q)}_{=\emptyset} \cup \underbrace{(B_\alpha \cap P)}_{=\emptyset} \cup \underbrace{(P \cap Q)}_{=\emptyset}\\
  &= A_\alpha \cap B_\alpha \in \Fin
 \end{align*}
 より $[A_{\alpha+1}] \wedge [B_{\alpha+1}] = 0$ となるのでOK.

 構成法より $A_{\alpha+1} \setminus A_{\alpha} = P,  B_{\alpha+1} \setminus B_\alpha = Q$ はいずれも無限集合なので,$[A_\alpha] < [A_{\alpha+1}], [B_\alpha] < [B_{\alpha+1}]$ である.帰納法の仮定より $[A_\beta] < [A_\alpha], [B_\beta] < [B_\alpha] \; (\beta < \alpha)$ が成立するので,これらを組み合わせれば $[A_\beta] < [A_{\alpha+1}], [B_\beta] < [B_{\alpha+1}] \; (\beta < \alpha+1)$ となり,条件 $()$ も成立.

 最後に $(d)$ が成立することを背理法により示そう.そこで,$\Set{ \beta < \alpha + 1 | A_{\alpha+1} \cap B_\beta \subseteq k }$ が無限となるような $k< \omega$ が存在したとする.この時,増大列 $\beta_n < \beta_{n+1} \; (n < \omega)$ であって $A_{\alpha + 1} \cap B_{\beta_n} \subseteq k$ となるものが取れる.構成から $A_\alpha \subseteq A_{\alpha+1}$ であるので,$A_\alpha \cap B_{\beta_n} \subseteq k \; (n < \omega)$ となる.これは帰納法の仮定に反する.よって $(d)$ も成立.
 以上より,$\alpha$ が後続順序数の時,条件 $(a) \sim (d)$ を満たすように $A_{\alpha}, B_\alpha$ を作ることが出来る.

 $\alpha = \beta$ が極限順序数の時.$\gamma < \beta$ のとき,帰納法の仮定の $(a)$ および $(c)$ と補題 1 から
 $[A_\gamma] \vee [B_\gamma] \leq [X] < 1 \; (\gamma < \beta)$
 を満たす $X \subseteq \omega$ が取れる.同様に補題より
 \begin{equation}
  [A_\gamma] \leq [S], \quad [B_\gamma] \wedge [S] = 0 \qquad (\gamma < \beta) \tag{1}
 \end{equation}
 を満たす $S$ が取れ,特に $S \subseteq X$ としてよい(特に $[X] \wedge [S]$ を考えれば,$[X] \wedge [S] \geq ([A_\gamma] \vee [B_\gamma]) \wedge [S] = [A_\gamma]$ であり,$[X] \wedge [S] \wedge [B_\gamma] = 0$ なので条件を満たす.また,$[X] \wedge [S] \subseteq [X]$ より $[S'] = [X] \wedge [S]$ で $S' \subseteq X$ を満たすような $S'$ が取れる).

 補題 2 を使って $A_\beta$ を定めたい.そこで,まずは $\beta = \omega$ の場合について,$[B_\gamma]$ について補題 $2$ の前提を満たす列 $S \subseteq [S_k]$ を作りたい:
\begin{equation}
   \begin{cases}
   [S_k] \leq [S_{k+1}] & (k < \omega)\\
   [B_n] \leq [B_{n+1}] & (n < \omega)\\
   [S_n] \wedge [B_n] = 0 & (n < \omega)
  \end{cases}\tag{2}
\end{equation}

 今,$I_k = \Set{ n < \omega | S \cap B_n \subseteq k} \; (k < \omega)$ とおき,これを用いて $S$ を膨らませた列を作ることを考える.上の条件を満たす $[S_k]$ を得るため,$[S_k] \wedge [B_n] = 0$ かつ $\Set{ n \in I_k | S_{k+1} \cap B_n \subseteq k}$ が有限となるように $[S_k]$ を帰納的に定める.$k = 0$ の時は,$S_0 = S$ とすれば良い.そこで,$S_k$ まで条件を満たすように構成出来たとして,$S_{k+1}$ を作ろう.

 $I_k$ が有限集合の時は,$S_{k + 1} = S_k$ とおく.$I_k$ が無限集合の時を考える.$\Set{ n < m | S \cap B_n \subseteq k}$ は有限集合であるので,$\braket{I_k, <}$ は各始切片が有限集合であるような無限整列集合である.このような性質を持つ順序数は $\omega$ のみであるので,同型 $e: \omega \rightarrow I_k$ が取れ,特に $e$ は狭義単調増加な全射である.更に,このとき $\sup \Set{e(n) | n < \omega} = \omega$ である.これを示すには,$e$ が全射であることから $\sup\Set{e(n) | n < \omega} = \sup I_k$ となるので,$\sup I_k = \omega$ を示せばよい.もし $\sup I_k = m < \omega$ とすれば,特に $I_k = \Set{ n < m + 1 | S \cap B_n \subseteq k}$ と書けることになる.今,$m + 1 < \omega$ であり, $(\ast)$ より $I_k$ は有限集合となり,仮定に反する.よって $\sup I_k = \omega$ となる.

 さて,$[B_\alpha]$ に関する帰納法の仮定 $(c)$ より $[B_n] < [B_{n+1}] \; (n < \omega)$ である.よって,数学的帰納法により $0 < [B_{e(n)} \setminus \bigcup_{i < n} B_{e(i)}] \leq [X]$ となることがわかる.従って $B_{e(n)} \setminus \bigcup_{i < n} B_{e(i)}$ が無限なので,$p_n \in (B_{e(n)} \setminus \bigcup_{i<n} B_{e(i)}) \cap X$ を満たすような $n \leq p_n$ が取れ,特に $p_n < p_{n+1}$ とできる.そこで,$S_{k+1} = \Set{p_k | k < \omega} \cup S_k$ と置く.この時,$B_{e(m)} \cap \Set{ p_n | n < \omega} \subseteq \Set{ p_n | n \leq m } \in \Fin$ より $[B_{e(m)}] \wedge [\Set{ p_n | n < \omega}] = 0$ であるので,帰納法の仮定と合わせて
 \begin{align*}
  [S_{k+1}] \wedge [B_{e(m)}] &= ([\Set{p_k | k < \omega}] \wedge [B_{e(m)}]) \vee ([S_k] \wedge [B_{e(m)}])\\
  &= 0 \vee 0 = 0 
 \end{align*}
 を得る.

 最後に $\ell < \omega$ について $\Set{n \in I_k | S_{k+1} \cap B_n \subseteq \ell}$ が有限であることを示す.まず,先程の議論より $e$ は $\omega$ から $I_k$ への順序同型なので $\Set{ n \in I_k | S_{k+1} \cap B_n \subseteq \ell} \approx \Set{ n < \omega | S_{k+1} \cap B_{e(n)} \subseteq \ell}$ である.今,$S_{k+1} \cap B_{e(n)} = (\Set{p_k | k < \omega} \cap B_{e(n)}) \cup (S_k \cap B_{e(n)})$ なので,これが $\subseteq \ell$ となるには,$\Set{p_k | k \leq n} \subseteq \ell$ となる必要があり,特に $p_n < \ell$ でなくてはならないが,$p_n$ の取り方より $n \leq p_n$ に取っているので,$n < \ell$ でなくてはならない.よって,$S_{k+1} \cap B_{e(n)} \subseteq \ell$ に含まれるような $n$ の候補は高々 $\ell$ 個しかない.よって,$\Set{ n \in I_k | S_{k+1} \cap B_n \subseteq \ell}$ は有限である.

 以上より,$(2)$ を満たすように $S_k \; (k < \omega)$ を取ることが出来た.よって,補題$2$ よりある $[A_\omega]$ が存在し,
 \[
  [S_k] \leq [A_\omega], [A_\omega] \wedge [B_n] = 0 \; (n < \omega)
 \]
 となる.特に,先程 $S$ を取った時と同様な議論により $A_\omega \subseteq X$ としてよい.よって,特に $[A_\omega] \leq [X] < 1 $ である.

 そこで,$B_\omega = X \setminus A_\omega$ とおいて,これが条件 $(a) \sim (d)$ を満たすことを示す.

 \begin{enumerate}[(a)]
  \item $[A_\omega] \vee [B_\omega] = [X] < 1$ なので成立.
  \item $[A_\omega] \wedge [B_\omega] = [\emptyset] = 0$ より成立.
  \item $n < \omega$ とすれば,帰納法の仮定により $[A_n] < [A_{n+1}] \leq [S_0] \leq [A_\omega]$ より $[A_n] < [A_\omega]$.また,$B_n \setminus B_\omega = B_n \cap A_n \in \Fin$ なので $[B_n] \leq [B_\omega]$.よって,先程と同様の議論により $[B_n] < [B_{n+1}] \leq B_\omega$ となる.よって OK.
  \item 任意の $k < \omega$ に対し, $\Set{n < \omega | A_\omega \cap B_n \subseteq k}$ が有限であることを示す.もし $I_k = \Set{ n < \omega | S \cap B_n \subseteq k}$ が有限であれば,$[S] \leq [A_n]$ であることから補題 $3$ が適用出来,$\Set{ n < \omega | A_\omega \cap B_n \subseteq k }$ も有限となる.

	そこで,$I_k$ が無限の場合を考える.この時,構成法から $\Set{n \in I_k | S_{k+1} \cap B_n \subseteq k}$ は有限である.よって,構成時に使った $e$ について,$\Set{n < \omega | S_{k+1} \cap B_{e(n)} \subseteq k}$ も有限.今,$[S_{k+1}] \leq [A_\omega]$ より,補題3から $\Set{n < \omega | A_\omega \cap B_{e(n)} \subseteq k}$ も有限となる.そこで,$n_0 = \sup^+ \Set{n < \omega | A_\omega \cap B_{e(n)} \subseteq k}$ とおけば $e(n_0) < \omega$ なので,$\Set{ n < e(n_0) | A_\omega \cap B_n \subseteq k}$ は有限となる.$n_0$ の取り方と $I_k$ の定義より,$\Set{n < \omega | A_\omega \cap B_n \subseteq k} = \Set{n < e(n_0) | A_\omega \cap B_n \subseteq k}$ となるので示された.
 \end{enumerate}

 最後に $\beta > \omega$ の場合を考える.$\omega < \beta < \omega_1$ より,$\beta$ は基数でないので特に特異順序数である.また,$\beta$ は可算な極限順序数であるので,$\cf(\beta) = \omega$ となる.そこで,$f: \omega \rightarrow \beta$ を狭義単調増加な共終写像とする.この時,$A'_n = A_{f(n)}, B'_n = B_{f(n)}$ を考えると,$A_\alpha, B_\alpha$ に関する帰納法の仮定から,上の議論を適用でき,$A'_\omega, B'_\omega$ が取れる.そこで $A_\beta = A'_\omega, B_\beta = B'_\omega$ とおけば,これが題意を満たすものとなっていることがわかる:$(a), (b)$ が成り立つことは明らか.$(c)$ については,$\alpha < \beta$ とすると,$\omega$ の $\beta$ での共終性から $n < \omega$ で $\alpha \leq f(n)$ となるものが取れる.よって $[A_\alpha] \leq [A_{f(n)}] < [A_\beta]$ となる.$[B_\beta]$ についても同様である.$(d)$ については,少し議論が必要である.まず,各 $k < \omega$ に対し $J_k = \Set{n < \omega | A_\beta \cap B_{f(n)} \subseteq k}$ は有限個である.そこで $n = \max J_k$ とおくと,$f$ の共終性と $B_n$ の単調性から $\Set{\alpha < \beta | A_\beta \cap B_\alpha \subseteq k} = \Set{ \alpha < f(n+1) | A_\beta \cap B_\alpha \subseteq k}$ となる.今,帰納法の仮定より $\Set{\alpha < f(n+1) | A_{f(n+1)} \cap B_\alpha \subseteq k}$ は有限.$f(n+1) < \beta$ より $[A_{f(n+1)}] = [A'_{n+1}] \leq [A'_\omega] = [A_\beta]$ であるので,補題 $3$ から $\Set{\alpha < f(n+1) | A_\beta \cap B_\alpha \subseteq k}$ も有限となる.以上より,任意の極限順序数 $\beta < \omega_1$ について必要な $A_\beta, B_\beta$ が構成出来る.

 以上より,$(a) \sim (d)$ を満たすような列 $A_\alpha, B_\alpha \; (\alpha < \omega_1)$ が取れた.そこで,$a_\alpha = [A_\alpha], b_\alpha = \neg [B_\alpha]$ とおけば,これが定理の主張する列となることを示す.

 まず,$a_\alpha < a_\beta, b_\beta < b_\alpha \; (\alpha < \beta)$ は条件 $(c)$ から直ちに従う.また,条件 $(b)$ より $a_\alpha \wedge \neg b_\alpha = [A_\alpha] \wedge [B_\alpha] = 0$ なので,ブール代数の一般論から $a_\alpha \leq b_\alpha$ となる.また,同様に条件 $(a)$ から $a_\alpha \vee \neg b_\alpha = [A_\alpha] \vee [B_\alpha] < 1$ なので $b_\alpha \not\leq a_\alpha$ である.よって $a_\alpha < b_\alpha \; (\alpha < \omega_1)$ となる.以上より $a_\alpha < a_\beta < b_\beta < b_\alpha \; (\alpha < \beta < \omega_1)$ は示された.

 二つめの条件を示せば,証明が完了する.そこで,$a_\alpha \leq b \leq b_\alpha \; (\alpha < \omega_1)$ となるような $b$ が存在したとして,矛盾を導こう.まず $\Set{\alpha < \omega_1 | B \cap B_\alpha \subseteq k}$が有限であることを示す.証明には,次の二つの命題を使う:

 \begin{prop}
  $\kappa:\text{正則基数}, X_\alpha \subseteq X_\beta \; (\alpha < \beta < \kappa)$ とする.この時,

  $\Set{X_\alpha | \alpha < \kappa} \text{に包含関係に関する最大元が存在しない} \Rightarrow \left|\bigcup \Set{X_\alpha | \alpha < \kappa}\right| \geq \kappa$
 \end{prop}
 \begin{proof}
  $\delta_0 = 0, \delta_\beta = \min \Set{\gamma < \kappa | X_\gamma \setminus \bigcup_{\alpha < \beta} X_{\delta_\alpha} \neq \emptyset} \; (\beta \neq 0)$ とおく.この時,任意の $\beta < \kappa$ に対し $\delta_\beta$ が定まる.もしある $\beta < \kappa$ に対し $\Set{\gamma < \kappa | X_\gamma \setminus \bigcup_{\alpha < \beta} X_{\delta_\alpha} \neq \emptyset} = \emptyset$  となったとすると,
 \[
  \forall \gamma < \kappa, \, X_\gamma \subseteq \bigcup \Set{X_{\delta_a} | \alpha < \beta}
 \]
 が成立する.今,$\kappa$ は正則なので,$\Set{\delta_\alpha | \alpha < \beta }$ は $\kappa$ で有界となる.よって,$\delta = \sup\Set{\delta_\alpha | \alpha < \beta } < \kappa$ が定まり,条件から $X_{\delta_\alpha} < X_\delta$ となる.すると,上の議論から $X_\gamma$ が $\Set{X_\alpha | \alpha < \kappa}$ の最大元となり矛盾.よって $\delta_\beta$ は well-defined である.そこで,$x_\beta \in X_{\delta_\beta} \setminus \bigcup \Set{X_{\delta_\alpha} | \alpha < \beta}$ を取れば,各 $x_\beta$ はそれぞれ異なるので,$\left|\Set{x_\beta | \beta < \kappa}\right| = \kappa$ である.よって $\Set{x_\beta | \beta < \kappa} \subseteq \bigcup \Set{X_\alpha | \alpha < \kappa}$ なので $\left|\bigcup \Set{X_\alpha | \alpha < \kappa}\right| \geq \kappa$ となる.\mbox{}
 \end{proof}

 更に,次の命題も成立する:
 \begin{prop}
  $\kappa:\text{基数}, F_\alpha : \text{有限集合}, (\alpha < \kappa),\; F_\alpha \subseteq F_\beta \; (\alpha < \beta < \kappa) \Longrightarrow \left|\bigcup \Set{F_\alpha | \alpha < \kappa}\right| \leq \omega$
 \end{prop}
 \begin{proof}
  まず,包含関係に関して正則基数型を持つ $\Set{F_\alpha | \alpha < \kappa}$ の共終部分集合を取る.共終性より,その共終部分集合の和集合は元の集合の和と一致するから,以後,$\kappa$ は正則基数だと思えばよい.

 そこで,命題$1$ に倣って
 \[
  \delta_0 = 0, \delta_\beta = \min \Set{\gamma < \kappa | X_\gamma \setminus \bigcup_{\alpha < \beta} X_{\delta_\alpha} \neq \emptyset} \; (\beta \neq 0)
 \]
とおき,$x_\beta$ を命題$1$と同様に定義する.$\delta_\beta$ が定義されるような $\beta$ の全体は明らかに順序数となるので,それを $\alpha$ と置く.この時,$\alpha \leq \omega$ である.もしこの $\alpha > \omega$ とすると,$\kappa > \omega$ であり,このとき $\Set{x_n | n < \omega} \subseteq F_\omega$ となってしまい,$F_i$ の有限性に反するからである.もし $\kappa \leq \omega$ ならば,可算集合の可算和は高々可算であることから主張は明らか.そこで,$\kappa > \omega$ とする.$\kappa$ は正則としてよかったので,$\delta = \sup^+ \delta_\alpha < \kappa$ が取れ,上の議論から特に $\delta \leq \omega$ となる.もし,$\delta = \omega$ とすると,$\delta_n$ の取り方より $F_{\delta_n} \subsetneq F_{\delta_m} \; (n < m)$ なので,$F_\omega$ が無限集合となり矛盾.よって,この場合は $\delta < \omega$ となるので,わかり易いように $N = \delta$ と書くことにする.このとき,$F_{\delta_n} \subsetneq F_\gamma \; (n < N)$ となるような $\gamma < \kappa$ が存在すれば,$F_\gamma \setminus \bigcup \Set{ F_{\delta_n} | n < N} \neq \emptyset$ なので,$\gamma = \delta_N$ となり矛盾.よって,$\Set{ F_{\delta_n} | n < N}$ は非有界なので,その和は元の集合の和に一致し,特に有限集合の有限和となるので,全体として有限になる.以上より,命題は示された.\mbox{}
 \end{proof}

 以上の二つの命題を踏まえて,$J_k = \Set{\alpha < \omega_1 | B \cap B_\alpha \subseteq k }$ の有限性を証明する.まず$A_\alpha, B_\alpha$ の構成法より,$\beta < \omega_1$ について,$\Set{\alpha < \beta | A_\beta \cap B_\alpha \subseteq k}$ は有限である.よって,補題$3$ および仮定の $[A_\beta] \leq [B]$ より $\Set{\alpha < \beta | B \cap B_\alpha \subseteq k}$ も有限となる.

 そこで,$F_\beta = \Set{\alpha < \beta | B \cap B_\alpha \subseteq k} \; (\beta < \omega_1)$ とおけば,$\Set{F_\beta | \beta < \kappa}$ は有限集合族であり,明らかに $F_\alpha \subseteq F_\beta \;(\alpha < \beta)$ となる.また,明らかに $J_k = \bigcup \Set{F_\alpha | \alpha < \omega_1}$ である.すると,命題 $2$ より $\left|\bigcup \Set{F_\alpha | \alpha < \kappa }\right| \leq \omega < \omega_1$ である.よって,$\omega_1$ の正則性と命題 $1$ の対偶より,$\Set{F_\alpha | \alpha < \omega_1}$ は最大元 $F_\gamma$ を持つ.よって,$F_\alpha \subseteq F_\gamma \;(\alpha < \omega_1)$ より $J_k = \bigcup \Set{F_\alpha | \alpha < \omega_1} = F_\gamma$ となる.$F_\gamma$ は有限だったから,各 $J_k$ も有限となる.

 すると,$\bigcup_{n < \omega} J_n$ は有限集合の可算和なので高々可算である.よって,$\alpha_0 \in \omega_1 \setminus \bigcup_{n < \omega} J_n$ が取れ,各 $J_k$ の定義より $B \cap B_{\alpha_0}$ は無限集合となる.よって,$b \wedge \neg [b_{\alpha_0}] = [B] \wedge [B_{\alpha_0}] > 0$ となるので,ブール代数の一般論より $b \not\leq b_{\alpha_0}$ となる.これは $b \leq b_\alpha$ に反する.よって,このような $b$ は存在しない.
 \qed
\end{proof}

\end{document}